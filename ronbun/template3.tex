\subsection{画像の挿入}
%\vspace{5pt}
画像を挿入したい時は,このファイルの最後にあるセットを組み込んでね.

\subsection{使用例}
%\vspace{5pt}
進捗が一向に進まない...そんな友人に送るべき画像をFig.\ref{fig:class}に示す.この画像を見せた時の反応は以下の5種類に分けられる.

\begin{itemize}
 \item 顔は笑いながらキレる
 \item 無表情でキレる
 \item 眉を吊り上げてキレる
 \item 無言でキレる
 \item 殴ってくる
\end{itemize}

画像を入れた際の改ページは注意が必要.改ページする前に全ての画像を出力したい時は,\ newpage \footnote{空白を開けているのはコマンドを実行して欲しく無いため}ではなく\ clearpage を使う.\ newpage を使用した場合,まだ出力されていない画像は改ページ後に出力される.

%%%%% ----- ここから画像挿入コマンド ----- %%%%%
\begin{figure}[tb]
\centering
\includegraphics[width=12cm]{fig/class.pdf} %ここに画像ファイルの名前を拡張子ごと入力,12cmは画像の大きさ
\caption{Class diagram} %題名はここ
\label{fig:class} %自由に名前をつけられる.図番号を呼び出す際に使う
\end{figure}