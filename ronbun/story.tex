\subsection{何故発展しないのか}

%\vspace{5pt}
\subsubsection{要因予想}
前章で拡張現実感の現状を述べた.少ないながらも拡張現実感を実現させるためのライブラリは提供されているが,何故ここまで拡張現実感は広まっていないのか?著者はその要因を以下のように予想した.

\begin{itemize}
 \item 興味を持つ人が少ない
 \item 拡張現実感が何かを知らない
 \item 拡張現実感を使う,という発想が無い
 \item 実装することで得られるメリットが少ない
 \item そもそも実装が難しい
\end{itemize}

ほかにも様々な要因があると思われるが,上に挙げただけでも拡張現実感が普及しない理由としては十分だと言えるだろう.
\par \vspace{5pt}
この要因に対して,解決案をそれぞれ考えることにした.

%\vspace{5pt}
\subsubsection{解決案}
先程述べた要因に対して,それぞれ解決案を考えてみた結果を以下に示す.
\begin{itemize}
 \item 興味を持つ人が少ない
 \item 拡張現実感が何かを知らない
       \begin{itemize}
	\item 拡張現実感を用いたイベント等で興味を持ってもらう
	      \begin{itemize}
	       \item 例) スマートフォンでかざしてみると様々な立体映像が見られる等
	      \end{itemize}
       \end{itemize}
       \vspace{10pt}
 \item 拡張現実感を使う,という発想が無い
 \item 実装することで得られるメリットが少ない
       \begin{itemize}
	\item 拡張現実感を用いることで得られるメリットをPRする
	      \begin{itemize}
	       \item 例) 拡張現実感を用いたシステム等からどれだけ付加価値があるか調べ,公表する
	      \end{itemize}
       \end{itemize}
 \item そもそも実装が難しい
       \begin{itemize}
	\item 開発しやすい環境を構築する
       \end{itemize}
\end{itemize}

\newpage
上記より,上から2つに関してはイベントなどで拡張現実感を使うというもので,まずは拡張現実感というものを知ってもらおうという案である.続く2つは,メリット等が分かっただけでは最後の``開発の難しさ''に直面し,これだけでは意味がない.
\par \vspace{5pt}
結果,まず取り組むべき要因は``拡張現実感に興味を持ってもらう''と``開発の難易度を下げる''だと考えた.

%\vspace{5pt}
\subsection{解決案}
上に挙げた要因より,``拡張現実感に興味を持ってもらう''と``開発の難易度を下げる''が最優先ではないかと考えた.そしてこの二つを解決するためには,以下の案がとても効果的ではないかという結論に至った.\par

\vspace{5pt}
\begin{itembox}[l]{解決案の提案}
拡張現実感を初めて体験する人に向けた簡易的なライブラリの制作
\end{itembox}
\par
\vspace{5pt}

また,使用するデバイスはAndroidを選択した.理由はカメラと画面が一体化しており,直感的な操作ができるから,また今後の発展性が大きく望めるからである.
\par \vspace{5pt}
しかし,プログラミング初心者向けの言語であるProcessing\cite{Processing}などと違い,Androidは複雑で高機能,どちらかといえばプログラミングに慣れた人でないと扱うのが難しい言語である.実際``Hello,World!''を表示する事はそこまで難しくないが,そこから新しい機能を追加するとなると途端に難しくなってしまう.ライブラリの追加方法やその利用方法などもある程度Android開発を経験している開発者向けの説明が殆どで,一から詳しく解説しているドキュメントはとても少ない.
\par \vspace{5pt}
Androidは拡張現実感を体験するにはとても良いが,開発が難しい.そこで,``とりあえず拡張現実感を体験してみよう''をテーマにしたAndroid向けライブラリの作成をしようと考えた.