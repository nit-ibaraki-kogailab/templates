本ライブラリによってAndroidアプリケーション開発初学者でも,比較的簡単に拡張現実感を体験する事ができるようになったと考える.ディスプレイを見ながらカメラを動かすのではなく,カメラの視点と自分の視点がほぼ同じで直感的な操作ができる点でAndroidはとても有用であり,今後の発展も望めるだろう.しかし課題も多くまだ実用的だとは言い難い.今後の発展で述べたもの以外にも,Raspberry Piとの連携など,更に拡張現実感の入門用ライブラリとしての十分な機能を実装したい.%\par
%\vspace{5pt}

拡張現実感ははじめに述べたように遊戯ではなく生活の補助として発展するだろうと予測している.車,テレビ,冷蔵庫などの登場,携帯電話の登場,スマートフォンの登場と,今までの生活が大きく変わる発明が過去に沢山あった.拡張現実感を用いた製品の登場が今までの生活を大きく変えるものとなるかどうかはまだ分からないが,大きく期待はしている.

%\par \vspace{5pt}
ヘッドマウントディスプレイ(HMD)と言うものがある.これは頭に装着して拡張現実感の出力先として使用する装置である.今はまだ機械自体が大きく,これを着けて外を歩くのは実用的ではない.しかし日本の得意分野である小型化によって簡単に外に持ち運べるようなものになった時,生活が大きく変わるだろうと考えている.また,スマートグラスという外見が眼鏡のようなものも数多く登場しており,拡張現実感の出力先の発展は遅れながらも確実に進んでいる.

%\par \vspace{5pt}
出力先の発展だけでは拡張現実感は発展しない.本研究のようなARライブラリの開発も必要不可欠である.今後,更にたくさんのARライブラリの開発が進むことを願っている.
最後に,本ライブラリによって更に拡張現実感に触れる機会が増え,興味を持っていただける事を期待している.