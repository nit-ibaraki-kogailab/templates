開発したライブラリに対しての考察をする.
%\vspace{5pt}
\subsection{考察1 : Setupの複雑化}
まず第一に,Setupクラスについてである.このクラスはライブラリ内では出来ない設定をするクラスであり,ライブラリと共に提供されると説明したが,設計初期はこのクラスは存在せず全てライブラリ内で初期設定をする予定だった.しかし設計と並行してライブラリの学習をしていた際,どうしてもライブラリでは出来ない設定があることを知り,当初無かったこのクラスを実装することになった.しかし設計時の予想よりもライブラリでできる事は少なく,結果Setupクラスが膨大になってしまった.Setupクラスを提供する事によって実装時の負担を減らすことが出来たが,分かりやすいライブラリからは少し遠ざかってしまった.後述するがここは今後の一番の課題であると考える.

%\vspace{5pt}
\subsection{考察2 : 拡張性}
第二に,拡張性についてである.前述したとおり,拡張性については達成できたとは言い難い.その原因として,簡易化するにあたって内部処理の大半をライブラリ内に格納してしまったからであると考える.これは拡張性と簡易化がトレード・オフの関係にあると言え,拡張性を求めると今度は簡易化が達成できない.しかし簡易化すると今度は拡張性が失われる.この二つを両立させることは難しいが,拡張する範囲を限定することで両立に近づけることはできるだろう.この場合どの範囲まで拡張性をもたせるかの判断も必要であり,後述する課題の一つと考えることができる.またライブラリ内の構成を工夫することでライブラリの改変を簡易化することができれば拡張性を失うことなく簡易かも達成できるだろうと考える.

%\vspace{5pt}
\subsection{考察3 : MainActivity}
第三に,MainActivityである.今回は親クラスを変更したが,MainActivityは所謂Mainクラスであるため,親クラスを変更することは望ましくない.既存のプログラムでMainActivityを既に改変していた際,本ライブラリの導入が出来ない可能性があるからだ.今回はARの体験,というコンセプトだった為,既存プログラムへの導入に関しては想定には含まなかったが,今後ライブラリを改良していく際の課題の一つであると考える.