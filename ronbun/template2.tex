\subsection{サブセクション名をここに書く}

%\vspace{5pt}
サブセクションの下にvspaceで余白を作る事で狭苦しくならない.
\subsubsection{さらにもう一段階作りたい時はこんなかんじ}
%\vspace{3pt}
サブサブの下に余白を入れるかどうかは好みによると思う.テンプレ作者は入れる派.これより細かいセクションはできない.そこまで細かくすると却って見づらくなってしまう.

\begin{itemize}
 \item 箇条書きはこんなかんじ
 \item いくらでも増やせるよ
       \begin{itemize}
	\item 入れ子もできるよ
	\item さすが\LaTeX だね
	      \begin{itemize}
	       \item ちなみに\TeX とか様々なコマンドが存在するよ
	       \item 特に数式のコマンドは人によってはたくさん使うんじゃないかな
	      \end{itemize}
	\item endで入れ子脱出.
       \end{itemize}
 \item 入れ子にしたらendを忘れずに.
\end{itemize}
画像の挿入は次のセクションで扱っているよ

\subsection{二つ目のセクション名}
%\vspace{5pt}
セクションは多ければ多いほど良いという訳では無いけれど,目次が豪華になってそれっぽく見えるよ.でもサブサブセクションは目次には反映されないよ.
%\par \vspace{5pt}

もしなんらかの理由(配置とか?)でページを変えたい場合は,
\newpage
で次のページに移るよ.画像を表示している時の改ページは注意点があるよ.それも次のセクションで扱うよ.
\par \vspace{5pt}
もし文章を枠で囲いたい時は...
\vspace{5pt}
\begin{itembox}[l]{ここに題名を書こう}
ここに内容を書こう
\end{itembox}
\par \vspace{5pt}
改行を忘れずに!

\subsection{追加説明したい時}
%\vspace{5pt}
なんらかの理由(配置とか?) ... あまりかっこよく無い!

%\par \vspace{5pt}
なんらかの理由\footnote{配置とか?} ... 論文っぽい!こっちの方がオススメ.