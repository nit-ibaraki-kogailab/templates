実装したライブラリの評価,課題及び今後の発展について,以下に述べる.
\vspace{5pt}
\subsection{評価}
実装したライブラリの検証,及び考察から,設計時のコンセプトは概ね実現していると言える.実際ARクラスを変更するだけで様々なBoxを表示することができるようになった.これによりARとはどんなものなのか,またAndroidでARを動かすために必要な処理がどんなものなのか等の学習に適切なライブラリになったと言えるだろう.またAndroudStudioの設定やライブラリの導入などを事前にしておけば中学生でもAndroidでARを動かすことができるだろう.

\vspace{5pt}
\subsection{課題}
検証及び考察で述べた通り,以下の課題が残っている.

\vspace{5pt}
\subsubsection{Setupの簡易化及びライブラリ内への格納}
先述した,今後一番の課題である.Setupクラスでは初学者には難しい初期設定などが記述されており,記述するにはとても難易度が高いものである.今回はライブラリと共に提供という形になったが,ライブラリ内に格納できればそのような問題はなくなる.この課題が達成できれば,Appモジュール内はMainActivity及びARクラスの二つのみとなり,使用者の作業が格段に簡易化される.ここまで簡易化されれば,プログラム未経験者対象のセミナーなどでARを体験することも容易になるだろう.

\vspace{5pt}
\subsubsection{拡張性の確保}
先述したが,拡張性は簡易化とトレード・オフの関係にありライブラリの簡易化は拡張性の喪失とも言える.しかしライブラリ内を更に簡易化し段階的に拡張性をもたせれば拡張性を喪失せずに簡易化できると考える.Appモジュール内の変更,ライブラリ内の変更,この二つを段階的にすれば拡張性をもたせつつ簡易化できるだろう.

\newpage
\subsubsection{MainActivityの親クラスの変更}
MainActivityの親クラスの変更は既存プログラムへの組み込みには適さない.この問題はSetupの簡易化及びライブラリ内への格納の問題が解決すると同じく解決するため,並行して進めていく予定である.

%\vspace{5pt}
\subsection{今後の発展}
本ライブラリを更に改良することで更に簡易化し,また拡張性をもたせ様々な人にとって使いやすいものにしようと考えている.また以下の機能を追加することで更にARについて学べるライブラリになると考えている.

\begin{itemize}
 \item 3Dオブジェクトの表示
 \item 時間経過による変化の組み込み
 \item 複数マーカの認識
 \item テキストの表示
 \item 画像,動画の表示
\end{itemize}

以上の機能を簡易化し,組み合わせやすくすることで今後の情報教育にも十分活用できるライブラリになると考える.