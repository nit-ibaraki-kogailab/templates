前章で実装したプログラムを実際に実行し,設定したコンセプトを達成できているか確かめる.
\vspace{5pt}
\subsection{関数名や変数名を明瞭化}
関数名や変数名を明瞭化する事によって拡張や改変する際にコードの役割が分かりやすくなる.
\begin{itemize}
	\item DrawやSetupなど,一目で何をするクラスなのかがよく分かるようになったと言える.ARクラスはそのままだと何をするクラスなのかが分かり辛いが,AR全般の設定をしている為このようなクラス名となった.
\end{itemize}
\vspace{5pt}
\subsection{各クラスの役割を明瞭化}
役割ごとにクラスを分ける事で,改変や拡張時にどこを変更するかがすぐに分かる.また改変時にエラーが生じた際,どこにエラーがあるかが分かりやすくなる.
\begin{itemize}
	\item MainActivity, Setup, AR, Draw, Cube等,クラスごとに役割を分け,何をするクラスなのかを明瞭化することが出来た.
\end{itemize}
\vspace{5pt}
\subsection{拡張しやすい構成に}
提供されたライブラリを使ってできることは限られており,またARを用いたAndroidプログラミングの学習用途であれば,拡張しやすい構成にすることは重要なことであると言える.
\begin{itemize}
	\item 拡張するにはライブラリ内を変更する必要があり,初学者には厳しいと考える.設計時の対象者では拡張は難しいだろう.拡張性については達成できたとは言い難い.
\end{itemize}
\vspace{5pt}
\subsection{MainActivityは最小限に}
MainActivityはCやJavaで言うMain関数である.Main関数は処理するクラスではなく,メソッドを実行するクラスであるため,ここでの処理は出来る限り最小限にすることが望ましい.
\begin{itemize}
	\item MainActivityはSetupクラスのsetupメソッドを呼び出すだけとなっている.その後はSetupクラスが初期設定をした後,ライブラリ内のARActivityが処理を引き継ぐ.
\end{itemize}