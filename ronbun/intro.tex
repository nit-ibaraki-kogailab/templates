2016年は「VR元年」とも呼ばれるほどにVRの発展が目覚ましかった.VR,すなわち仮想現実感は被験者に完全な仮想の視界を提供する技術であり,被験者の視界は完全に遮られ代わりに新しい視界を与えられる\cite{VR}.
今まで以上にリアリティの高い経験を得ることができることから世界中の注目を集め,今では一つの分野として成長を続けている.

%\par
%\vspace{5pt}
また,VRと似た分野にARというものがある.AR,すなわち拡張現実感とは,被験者の視界を完全には遮らず,普段の視界を保ったまま新たな情報を提供する技術である.拡張現実感と仮想現実感は名前こそ似ているがその実態は全く異なり,使われている技術や想定されている使用用途など様々な点で違いがある.

%\par
%\vspace{5pt}
仮想現実感は視界を完全に遮るため自身の移動に大きな制限が掛かってしまう.その為殆どの場合その場から動いて使用するような想定はされていない.これは据え置き型のゲームとの相性がとても良く,現在も仮想現実感を用いた据え置き型ゲームの市場は拡大しつつあると言える\cite{VRnews}.

%\par
%\vspace{5pt}
拡張現実感は仮想現実感とは違い現実の視界が保たれているため,移動しながらの使用も可能である.また移動しながら使用することを想定したアプリケーションも開発されてきた\cite{ARapp}.

%\par
%\vspace{5pt}
ここで仮想現実感と拡張現実感の違いのもう一つ,すなわち使われている技術について注目してみると,意外なほどに異なっている.仮想現実感では3Dグラフィックス,頭の向きを測るための加速度計測などが中心に使われる.しかし拡張現実感では3Dグラフィックスに加え画像認識を用いている.つまり向いている方向を判断する技術が全く異なるのだ.

%\par
%\vspace{5pt}
今までに挙げた仮想現実感と拡張現実感の特徴をまとめると,仮想現実感は趣味や娯楽方面でとても有用だということがわかる.また拡張現実感は仕事や生活の効率化に有用だということがわかる.しかし,拡張現実感は仮想現実感に比べ世間の注目度や浸透度などにおいて遅れをとっていると言える.

%\par
%\vspace{5pt}
何故,拡張現実感はそこまで仮想現実感に比べ発展していないのだろうか?著者はこの疑問に対する回答及び解決案を提示することにした.