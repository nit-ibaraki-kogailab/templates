%\vspace{10pt}
\subsection{コードの埋め込み}
もしコードを埋め込みたいなら...

\begin{lstlisting}
#include <stdio.h>
int main(void)
{
	printf("Hello, World !\n");
	return 0;
}
\end{lstlisting}
この中では改行も含めて書いたままが出力されるよ.

\subsection{表記の統一}
拡張現実と拡張現実感など,表記ゆれに気をつける.気をつけると気を付けるなど漢字にも注意.二日間と2日間もよくある表記ゆれ.また語尾は~だ,~である調で書く.\footnote{ただし謝辞は~です~ます調で書かれることが多い.}

\subsection{長音符の扱い}
四字以上で最後が長音符だった時,長音符は省略する.\\例
\begin{itemize}
 \item NG:プリンター OK:プリンタ
 \item NG:キャラクター OK:キャラクタ
 \item NG:メンバー OK:メンバ
\end{itemize}
この間違いは多発するので特に注意.

\subsection{句読点}
句読点の打ちすぎに注意.例えば,こんな感じで,多用すると,とても,読みづらい.\\
読点が多くなったら文の前後入れ替えや語尾の変更などでいい感じに整えよう.

\newpage
ここから先はテンプレート作成者田村の論文です.参考になるかは分かりませんが...\\
今まで1年間研究を頑張ってきたと思いますが,論文作成は1年間の総仕上げです.これからが本番と言っても過言ではないでしょう.論文は計画的に.2日前には担当教員に提出できるように!\footnote{ちなみに作成者は提出日に寝坊して3時間遅れで提出しました}

%\par \vspace{5pt}
最後の謝辞に少しだけ注釈を入れてあります.消去するのを忘れないように気をつけて下さい.
%\par
%\vspace{10pt}

作成日 2017年 2月 27日\footnote{2月28日が卒研発表日でした} 小飼研 田村 優